\chapter{Implementacja}

\section{Architektura}

Diagram \ref{fig:architektura} przedstawia architekturę aplikacji wraz z jej osadzeniem w potencjalnym środowisku uruchomieniowym.

Aplikacja składa się z 3 modułów:

\begin{itemize}
	\item Systemu udostępniającego zadania do wykonania. 
	Jego zadaniem jest generowanie, zlecanie oraz prezentowanie uzyskanych wyników planowania.
	\item Systemu odbierającego zadania planowania i przekazującego żądania do węzłów obliczeniowych.
	\item Węzła obliczeniowego wykonującego planowanie przy pomocy wybranych algorytmów. 
	Jego zadaniem jest wykonanie planowania według zadanych parametrów, zebranie informacji związanych z efektywnością (np. czas wykonania, ilość wykorzystywanej pamięci etc.) wybranego algorytmu, oraz zwrócenie wyników do zlecającego klienta.
\end{itemize}

W ramach konkretyzowania modelu przedstawionego w koncepcji na rysunku \ref{fig:wizja} zdecydowaliśmy się wykorzystać SOAP WebService w celu przyjmowania zadań od klientów.
Po stronie naszej referencyjnej implementacji klienta zdecydowaliśmy się wykorzystac REST WebService w celu odbierania wyników planowania.
W sekcji \ref{sec:webservices} przedstawiamy krótką motywację tej decyzji.
Dodatkowo zdecydowaliśmy się na konkretną reprezentację zadań planowania oraz ich parametrów - postanowiliśmy wykorzystać reprezentację GEXF, a także na reprezentację wyników.
Obie te reprezenatcje oraz motywację dla tej decyzji przedstawiliśmy w sekcji \ref{sec:data}.

\begin{figure}
	\centering
	\includegraphics[width=\linewidth]{img/architektura}
	\caption{Architektura aplikacji.}
	\label{fig:architektura}
\end{figure}

\section{Web serwisy}
\label{sec:webservices}

\subsection{Web serwis przyjmujący zadania}
\paragraph{}
Centralną częścią naszego projektu jest Web Service przyjmujący zadania planowania. 
Przyjmuje on wiadomości przesyłanie w formacie SOAP. 
Do jego zaimplementowania wykorzystaliśmy bibliotekę Apache CXF.
Poniżej znajduje się definicja naszego web serwisu:

\begin{program}
\begin{code}
\begin{Verbatim}[commandchars=\\\{\}]
\PY{n+nt}{\PYZlt{}wsdl:portType} \PY{n+na}{name=}\PY{l+s}{\PYZdq{}GraphPlanningPortType\PYZdq{}}\PY{n+nt}{\PYZgt{}}
  \PY{n+nt}{\PYZlt{}wsdl:operation} \PY{n+na}{name=}\PY{l+s}{\PYZdq{}schedule\PYZhy{}planning\PYZdq{}}\PY{n+nt}{\PYZgt{}}
    \PY{n+nt}{\PYZlt{}wsdl:input} \PY{n+na}{message=}\PY{l+s}{\PYZdq{}tns:planning\PYZhy{}task\PYZdq{}} \PY{n+na}{name=}\PY{l+s}{\PYZdq{}planning\PYZhy{}task\PYZdq{}}\PY{n+nt}{/\PYZgt{}}
    \PY{n+nt}{\PYZlt{}wsdl:output} \PY{n+na}{message=}\PY{l+s}{\PYZdq{}tns:planning\PYZhy{}task\PYZhy{}response\PYZdq{}} 
      \PY{n+na}{name=}\PY{l+s}{\PYZdq{}planning\PYZhy{}task\PYZhy{}response\PYZdq{}}\PY{n+nt}{/\PYZgt{}}
  \PY{n+nt}{\PYZlt{}/wsdl:operation\PYZgt{}}
\PY{n+nt}{\PYZlt{}/wsdl:portType\PYZgt{}}
\end{Verbatim}
\end{code}
\caption{PortType Web Service'u przyjmującego zadania}
\end{program}

\begin{program}
\begin{code}
\begin{Verbatim}[commandchars=\\\{\}]
\PY{n+nt}{\PYZlt{}wsdl:message} \PY{n+na}{name=}\PY{l+s}{\PYZdq{}planning\PYZhy{}task\PYZdq{}}\PY{n+nt}{\PYZgt{}}
    \PY{n+nt}{\PYZlt{}wsdl:part} \PY{n+na}{element=}\PY{l+s}{\PYZdq{}tns:planning\PYZhy{}task\PYZdq{}} \PY{n+na}{name=}\PY{l+s}{\PYZdq{}parameters\PYZdq{}}\PY{n+nt}{/\PYZgt{}}
\PY{n+nt}{\PYZlt{}/wsdl:message\PYZgt{}}
\PY{n+nt}{\PYZlt{}wsdl:message} \PY{n+na}{name=}\PY{l+s}{\PYZdq{}planning\PYZhy{}task\PYZhy{}response\PYZdq{}}\PY{n+nt}{\PYZgt{}}
    \PY{n+nt}{\PYZlt{}wsdl:part} \PY{n+na}{element=}\PY{l+s}{\PYZdq{}tns:planning\PYZhy{}task\PYZhy{}response\PYZdq{}} \PY{n+na}{name=}\PY{l+s}{\PYZdq{}parameters\PYZdq{}}\PY{n+nt}{/\PYZgt{}}
\PY{n+nt}{\PYZlt{}/wsdl:message\PYZgt{}}
\end{Verbatim}
\end{code}
\caption{Zapytanie oraz odpowiedź serwisu}
\end{program}

\begin{program}
\begin{code}
\begin{Verbatim}[commandchars=\\\{\}]
\PY{n+nt}{\PYZlt{}xs:complexType} \PY{n+na}{name=}\PY{l+s}{\PYZdq{}planning\PYZhy{}task\PYZdq{}}\PY{n+nt}{\PYZgt{}}
  \PY{n+nt}{\PYZlt{}xs:sequence}\PY{n+nt}{\PYZgt{}}
      \PY{n+nt}{\PYZlt{}xs:element} \PY{n+na}{minOccurs=}\PY{l+s}{\PYZdq{}1\PYZdq{}} \PY{n+na}{maxOccurs=}\PY{l+s}{\PYZdq{}1\PYZdq{}} \PY{n+na}{name=}\PY{l+s}{\PYZdq{}graph\PYZdq{}} 
	\PY{n+na}{type=}\PY{l+s}{\PYZdq{}gexf:gexf\PYZhy{}content\PYZdq{}}\PY{n+nt}{/\PYZgt{}}
      \PY{n+nt}{\PYZlt{}xs:element} \PY{n+na}{minOccurs=}\PY{l+s}{\PYZdq{}1\PYZdq{}} \PY{n+na}{maxOccurs=}\PY{l+s}{\PYZdq{}1\PYZdq{}} \PY{n+na}{name=}\PY{l+s}{\PYZdq{}algorithm\PYZdq{}} 
	\PY{n+na}{type=}\PY{l+s}{\PYZdq{}tns:planning\PYZhy{}algorithm\PYZdq{}}\PY{n+nt}{/\PYZgt{}}
      \PY{n+nt}{\PYZlt{}xs:element} \PY{n+na}{minOccurs=}\PY{l+s}{\PYZdq{}1\PYZdq{}} \PY{n+na}{maxOccurs=}\PY{l+s}{\PYZdq{}1\PYZdq{}} \PY{n+na}{name=}\PY{l+s}{\PYZdq{}response\PYZhy{}service\PYZdq{}} 
	\PY{n+na}{type=}\PY{l+s}{\PYZdq{}tns:response\PYZhy{}service\PYZdq{}}\PY{n+nt}{/\PYZgt{}}
  \PY{n+nt}{\PYZlt{}/xs:sequence\PYZgt{}}
\PY{n+nt}{\PYZlt{}/xs:complexType\PYZgt{}}
\end{Verbatim}
\end{code}
\caption{Zadanie planowania}
\end{program}

\begin{program}
\begin{code}
\begin{Verbatim}[commandchars=\\\{\}]
\PY{n+nt}{\PYZlt{}xs:complexType} \PY{n+na}{name=}\PY{l+s}{\PYZdq{}planning\PYZhy{}task\PYZhy{}response\PYZdq{}}\PY{n+nt}{\PYZgt{}}
    \PY{n+nt}{\PYZlt{}xs:sequence}\PY{n+nt}{\PYZgt{}}
	\PY{n+nt}{\PYZlt{}xs:element} \PY{n+na}{minOccurs=}\PY{l+s}{\PYZdq{}0\PYZdq{}} \PY{n+na}{name=}\PY{l+s}{\PYZdq{}status\PYZdq{}} \PY{n+na}{type=}\PY{l+s}{\PYZdq{}tns:response\PYZhy{}status\PYZdq{}}\PY{n+nt}{/\PYZgt{}}
	\PY{n+nt}{\PYZlt{}xs:element} \PY{n+na}{minOccurs=}\PY{l+s}{\PYZdq{}0\PYZdq{}} \PY{n+na}{name=}\PY{l+s}{\PYZdq{}job\PYZhy{}id\PYZdq{}} \PY{n+na}{type=}\PY{l+s}{\PYZdq{}xs:string\PYZdq{}}\PY{n+nt}{/\PYZgt{}}
    \PY{n+nt}{\PYZlt{}/xs:sequence\PYZgt{}}
\PY{n+nt}{\PYZlt{}/xs:complexType\PYZgt{}}
\end{Verbatim}
\end{code}
\caption{Odpowiedź serwisu}
\end{program}

\begin{program}
\begin{code}
\begin{Verbatim}[commandchars=\\\{\}]
\PY{n+nt}{\PYZlt{}xs:simpleType} \PY{n+na}{name=}\PY{l+s}{\PYZdq{}response\PYZhy{}status\PYZdq{}}\PY{n+nt}{\PYZgt{}}
  \PY{n+nt}{\PYZlt{}xs:restriction} \PY{n+na}{base=}\PY{l+s}{\PYZdq{}xs:string\PYZdq{}}\PY{n+nt}{\PYZgt{}}
    \PY{n+nt}{\PYZlt{}xs:enumeration} \PY{n+na}{value=}\PY{l+s}{\PYZdq{}OK\PYZdq{}}\PY{n+nt}{/\PYZgt{}}
    \PY{n+nt}{\PYZlt{}xs:enumeration} \PY{n+na}{value=}\PY{l+s}{\PYZdq{}UNSUPPORTED\PYZdq{}}\PY{n+nt}{/\PYZgt{}}
  \PY{n+nt}{\PYZlt{}/xs:restriction\PYZgt{}}
\PY{n+nt}{\PYZlt{}/xs:simpleType\PYZgt{}}
\end{Verbatim}
\end{code}
\caption{Status zadania}
\end{program}

\begin{program}
\begin{code}
\begin{Verbatim}[commandchars=\\\{\}]
\PY{n+nt}{\PYZlt{}xs:complexType} \PY{n+na}{name=}\PY{l+s}{\PYZdq{}response\PYZhy{}service\PYZdq{}}\PY{n+nt}{\PYZgt{}}
  \PY{n+nt}{\PYZlt{}xs:sequence}\PY{n+nt}{\PYZgt{}}
    \PY{n+nt}{\PYZlt{}xs:element} \PY{n+na}{minOccurs=}\PY{l+s}{\PYZdq{}1\PYZdq{}} \PY{n+na}{name=}\PY{l+s}{\PYZdq{}url\PYZdq{}} \PY{n+na}{type=}\PY{l+s}{\PYZdq{}xs:string\PYZdq{}}\PY{n+nt}{/\PYZgt{}}
    \PY{n+nt}{\PYZlt{}xs:element} \PY{n+na}{minOccurs=}\PY{l+s}{\PYZdq{}1\PYZdq{}} \PY{n+na}{name=}\PY{l+s}{\PYZdq{}method\PYZdq{}} \PY{n+na}{type=}\PY{l+s}{\PYZdq{}tns:response\PYZhy{}method\PYZdq{}}\PY{n+nt}{/\PYZgt{}}
  \PY{n+nt}{\PYZlt{}/xs:sequence\PYZgt{}}
\PY{n+nt}{\PYZlt{}/xs:complexType\PYZgt{}}
\end{Verbatim}
\end{code}
\caption{Serwis przyjmujący odpowiedzi}
\end{program}

\begin{program}
\begin{code}
\begin{Verbatim}[commandchars=\\\{\}]
\PY{n+nt}{\PYZlt{}xs:simpleType} \PY{n+na}{name=}\PY{l+s}{\PYZdq{}response\PYZhy{}method\PYZdq{}}\PY{n+nt}{\PYZgt{}}
  \PY{n+nt}{\PYZlt{}xs:restriction} \PY{n+na}{base=}\PY{l+s}{\PYZdq{}xs:string\PYZdq{}}\PY{n+nt}{\PYZgt{}}
    \PY{n+nt}{\PYZlt{}xs:enumeration} \PY{n+na}{value=}\PY{l+s}{\PYZdq{}POST\PYZdq{}}\PY{n+nt}{/\PYZgt{}}
  \PY{n+nt}{\PYZlt{}/xs:restriction\PYZgt{}}
\PY{n+nt}{\PYZlt{}/xs:simpleType\PYZgt{}}
\end{Verbatim}
\end{code}
\caption{Metoda zwracania odpowiedzi}
\end{program}

\begin{program}
\begin{code}
\begin{Verbatim}[commandchars=\\\{\}]
\PY{n+nt}{\PYZlt{}xs:simpleType} \PY{n+na}{name=}\PY{l+s}{\PYZdq{}planning\PYZhy{}algorithm\PYZdq{}}\PY{n+nt}{\PYZgt{}}
  \PY{n+nt}{\PYZlt{}xs:restriction} \PY{n+na}{base=}\PY{l+s}{\PYZdq{}xs:string\PYZdq{}}\PY{n+nt}{\PYZgt{}}
    \PY{n+nt}{\PYZlt{}xs:enumeration} \PY{n+na}{value=}\PY{l+s}{\PYZdq{}AStar\PYZdq{}}\PY{n+nt}{/\PYZgt{}}
    \PY{n+nt}{\PYZlt{}xs:enumeration} \PY{n+na}{value=}\PY{l+s}{\PYZdq{}BellmanFord\PYZdq{}}\PY{n+nt}{/\PYZgt{}}
    \PY{n+nt}{\PYZlt{}xs:enumeration} \PY{n+na}{value=}\PY{l+s}{\PYZdq{}BFS\PYZdq{}}\PY{n+nt}{/\PYZgt{}}
    \PY{n+nt}{\PYZlt{}xs:enumeration} \PY{n+na}{value=}\PY{l+s}{\PYZdq{}DFS\PYZdq{}}\PY{n+nt}{/\PYZgt{}}
    \PY{n+nt}{\PYZlt{}xs:enumeration} \PY{n+na}{value=}\PY{l+s}{\PYZdq{}Dijkstra\PYZdq{}}\PY{n+nt}{/\PYZgt{}}
    \PY{n+nt}{\PYZlt{}xs:enumeration} \PY{n+na}{value=}\PY{l+s}{\PYZdq{}FloydWarshall\PYZdq{}}\PY{n+nt}{/\PYZgt{}}
    \PY{n+nt}{\PYZlt{}xs:enumeration} \PY{n+na}{value=}\PY{l+s}{\PYZdq{}GreedyBestFirst\PYZdq{}}\PY{n+nt}{/\PYZgt{}}
    \PY{n+nt}{\PYZlt{}xs:enumeration} \PY{n+na}{value=}\PY{l+s}{\PYZdq{}Kruskal\PYZdq{}}\PY{n+nt}{/\PYZgt{}}
    \PY{n+nt}{\PYZlt{}xs:enumeration} \PY{n+na}{value=}\PY{l+s}{\PYZdq{}Prim\PYZdq{}}\PY{n+nt}{/\PYZgt{}}
    \PY{n+nt}{\PYZlt{}xs:enumeration} \PY{n+na}{value=}\PY{l+s}{\PYZdq{}UniformCost\PYZdq{}}\PY{n+nt}{/\PYZgt{}}
  \PY{n+nt}{\PYZlt{}/xs:restriction\PYZgt{}}
\PY{n+nt}{\PYZlt{}/xs:simpleType\PYZgt{}}
\end{Verbatim}
\end{code}
\caption{Algorytmy planowania}
\end{program}


\begin{program}
\begin{code}
\begin{Verbatim}[commandchars=\\\{\}]
\PY{n+nt}{\PYZlt{}wsdl:binding} \PY{n+na}{name=}\PY{l+s}{\PYZdq{}GraphPlanningPortTypeSoapBinding\PYZdq{}} 
  \PY{n+na}{type=}\PY{l+s}{\PYZdq{}tns:GraphPlanningPortType\PYZdq{}}\PY{n+nt}{\PYZgt{}}
  \PY{n+nt}{\PYZlt{}soap:binding} \PY{n+na}{style=}\PY{l+s}{\PYZdq{}document\PYZdq{}} 
    \PY{n+na}{transport=}\PY{l+s}{\PYZdq{}http://schemas.xmlsoap.org/soap/http\PYZdq{}}\PY{n+nt}{/\PYZgt{}}
  \PY{n+nt}{\PYZlt{}wsdl:operation} \PY{n+na}{name=}\PY{l+s}{\PYZdq{}schedule\PYZhy{}planning\PYZdq{}}\PY{n+nt}{\PYZgt{}}
    \PY{n+nt}{\PYZlt{}soap:operation} \PY{n+na}{soapAction=}\PY{l+s}{\PYZdq{}\PYZdq{}} \PY{n+na}{style=}\PY{l+s}{\PYZdq{}document\PYZdq{}}\PY{n+nt}{/\PYZgt{}}
    \PY{n+nt}{\PYZlt{}wsdl:input} \PY{n+na}{name=}\PY{l+s}{\PYZdq{}planning\PYZhy{}task\PYZdq{}}\PY{n+nt}{\PYZgt{}}
      \PY{n+nt}{\PYZlt{}soap:body} \PY{n+na}{use=}\PY{l+s}{\PYZdq{}literal\PYZdq{}}\PY{n+nt}{/\PYZgt{}}
    \PY{n+nt}{\PYZlt{}/wsdl:input\PYZgt{}}
    \PY{n+nt}{\PYZlt{}wsdl:output} \PY{n+na}{name=}\PY{l+s}{\PYZdq{}planning\PYZhy{}task\PYZhy{}response\PYZdq{}}\PY{n+nt}{\PYZgt{}}
      \PY{n+nt}{\PYZlt{}soap:body} \PY{n+na}{use=}\PY{l+s}{\PYZdq{}literal\PYZdq{}}\PY{n+nt}{/\PYZgt{}}
    \PY{n+nt}{\PYZlt{}/wsdl:output\PYZgt{}}
  \PY{n+nt}{\PYZlt{}/wsdl:operation\PYZgt{}}
\PY{n+nt}{\PYZlt{}/wsdl:binding\PYZgt{}}
\end{Verbatim}
\end{code}
\caption{Binding Web Service'u}
\end{program}



\begin{program}
\begin{code}
\begin{Verbatim}[commandchars=\\\{\}]
\PY{n+nt}{\PYZlt{}wsdl:service} \PY{n+na}{name=}\PY{l+s}{\PYZdq{}GraphPlanningPortTypeService\PYZdq{}}\PY{n+nt}{\PYZgt{}}
  \PY{n+nt}{\PYZlt{}wsdl:port} \PY{n+na}{binding=}\PY{l+s}{\PYZdq{}tns:GraphPlanningPortTypeSoapBinding\PYZdq{}}
\PY{n+na}{name=}\PY{l+s}{\PYZdq{}GraphPlanningServicePort\PYZdq{}}\PY{n+nt}{\PYZgt{}}
    \PY{n+nt}{\PYZlt{}soap:address} \PY{n+na}{location=}\PY{l+s}{\PYZdq{}http://localhost:9000/planning\PYZdq{}}\PY{n+nt}{/\PYZgt{}}
  \PY{n+nt}{\PYZlt{}/wsdl:port\PYZgt{}}
\PY{n+nt}{\PYZlt{}/wsdl:service\PYZgt{}}
\end{Verbatim}
\end{code}
\caption{Implementacja serwisu przyjmującego zadania}
\end{program}


\subsection{Web serwis przyjmujący wyniki}
\paragraph{}
Web Service przyjmujący wyniki 

\section{Format danych}
\label{sec:data}

\subsection{GEXF}

\subsection{JSON}

\section{Algorytmy}

\subsection{BFS}
\paragraph{}
Przeszukiwanie wszerz (Breadth-first search) to jeden z najprostszych algorytmów przeszukiwania grafu. 
Przechodzenie grafu rozpoczyna się od zadanego wierzchołka i polega na odwiedzeniu wszystkich osiągalnych z niego wierzchołków. 
Wynikiem działania algorytmu jest także drzewo przeszukiwania wszerz o korzeniu w zadanym wierzchołku, zawierające wszystkie wierzchołki do których
prowadzi droga z wierzchołka początkowego. 
Algorytm działa prawidłowo zarówno dla grafów skierowanych jak i nieskierowanych.
Nie ma jednak możliwości zastosowania go do wyznaczenia ścieżki w grafie ważonym.

\paragraph{}
Na rysunku \ref{fig:bfs} przedstawiony został wynik działania algorytmu BFS na przykładowym grafie nieważonym.

\begin{figure}[!h]
 \centering
 \includegraphics[width=1.0\textwidth]{algorithms/bfs}
 \caption{Wynik działania algorytmu BFS na przykładowym skierowanym, nieważonym grafie}
 \label{fig:bfs}
\end{figure}

\subsection{DFS}
\paragraph{}
Przeszukiwanie w głąb (Depth-first search) to również jeden z najprostszych algorytmów przeszukiwania grafu. 
Polega on na badaniu wszystkich krawędzi wychodzących z podanego wierzchołka w taki sposób, aby najpierw wybrać jeden z bezpośrednio przyległych
wierzchołków i zagłębić się w nim.
Po zbadaniu wszystkich krawędzi wychodzących z danego wierzchołka algorytm powraca do wierzchołka, z którego dany wierzchołek został odwiedzony.
\paragraph{}
Algorytm ten działa prawidłowo jedynie dla grafów nieskierowanych i nieważonych.

\paragraph{}
Rysunek \ref{fig:dfs} przedstawia wynik działania algorytmu przeszukiwania w głąb na nieskierowanym, nieważonym grafie.

\begin{figure}[!h]
 \centering
 \includegraphics{algorithms/dfs}
 \caption{Wynik działania algorytmu DFS na przykładowym nieskierowanym, nieważonym grafie}
 \label{fig:dfs}
\end{figure}

\subsection{A*}
\paragraph{}
Algorytm A* to heurystyczny algorytm znajdowania najkrótszej ścieżki w grafie ważonym z dowolnego wierzchołka do wierzchołka spełniającego określony
warunek zwany testem celu. 
Algorytm jest zupełny i optymalny, w tym sensie, że znajduje ścieżkę, jeśli tylko taka istnieje, i przy tym jest to ścieżka najkrótsza. 
Stosowany głównie w dziedzinie sztucznej inteligencji do rozwiązywania problemów i w grach komputerowych do imitowania inteligentnego zachowania.
Algorytm został opisany pierwotnie w 1968 roku przez Petera Harta, Nilsa Nilssona oraz Bertrama Raphaela. 
W ich pracy naukowej został nazwany algorytmem A. 
Ponieważ jego użycie daje optymalne zachowanie dla danej heurystyki, nazwano go A*.

\paragraph{}
Na rysunku \ref{fig:aStar} przedstawiony został wynik działania algorytmu A* na przykładowym grafie.

\begin{figure}[!h]
 \centering
 \includegraphics{algorithms/dfs}
 \caption{Wynik działania algorytmu A* na przykładowym skierowanym, ważonym grafie}
 \label{fig:aStar}
\end{figure}

\subsection{Bellman-Ford}
\paragraph{}
Algorytm Bellmana-Forda pozwala znaleźć ścieżkę o najmniejszej wadze pomiędzy zadanym wierzchołkiem początkowym a wszystkimi pozostałymi
wierzchołkami w grafie ważonym.
Idea algorytmu opiera się na metodzie relaksacji (następuje relaksacja V-1 razy każdej z krawędzi).
Algorytm ten pozwala znaleźć najkrótsze ścieżki również w grafie ważonym z ujemnymi wagami. 
Nie może jednak występować cykl o łącznej ujemnej wadze osiągalny ze źródła. 
Za tę ogólność płaci się jednak wyższą złożonością czasową. 
Działa on w czasie O(V*E).
Na algorytmie Bellmana-Forda bazuje protokół RIP - Routing Information Protocol.

\paragraph{}
Jak widać na rysunku \ref{fig:bellmanFord}, algorytm Bellmana-Forda wyznaczył poprawną ścieżkę w zadanym grafie.

\begin{figure}[!h]
 \centering
 \includegraphics{algorithms/bellmanFord}
 \caption{Wynik działania algorytmu Bellmana-Forda na przykładowym skierowanym, ważonym grafie}
 \label{fig:bellmanFord}
\end{figure}

\subsection{Dijkstra}
\paragraph{}
Algorytm Dijkstry opracowany został przez holenderskiego informatyka Edsgera Dijkstrę.
Służy on do znajdowania najkrótszej ścieżki z pojedynczego źródła w grafie o nieujemnych wagach krawędzi.
Mając dany graf z wyróżnionym wierzchołkiem (źródłem) algorytm znajduje odległości od źródła do wszystkich pozostałych wierzchołków. 
Łatwo zmodyfikować go tak, aby szukał wyłącznie (najkrótszej) ścieżki do jednego ustalonego wierzchołka, po prostu przerywając działanie w momencie
dojścia do wierzchołka docelowego, bądź transponując tablicę incydencji grafu.
Algorytm Dijkstry znajduje w grafie wszystkie najkrótsze ścieżki pomiędzy wybranym wierzchołkiem a wszystkimi pozostałymi, przy okazji wyliczając
również koszt przejścia każdej z tych ścieżek.
Algorytm Dijkstry jest przykładem algorytmu zachłannego.


\paragraph{}
Jak widać na rysunku \ref{fig:dijkstra}, algorytm Dijkstry wyznaczył poprawną ścieżkę w zadanym grafie.

\begin{figure}[!h]
 \centering
 \includegraphics{algorithms/dijkstra}
 \caption{Wynik działania algorytmu Dijkstry na przykładowym skierowanym, ważonym grafie}
 \label{fig:dijkstra}
\end{figure}

\subsection{Uniform Cost}
\paragraph{}
Algorytm Uniform-cost wykorzystywany jest do przeszukiwania struktur drzewiastych oraz grafowych.
Opiera się on, na wybieraniu tego wierzchołka do odwiedzenia, który ma najmniejszą łączną sumę wag na zadanej ścieżce.

\paragraph{}
Na rysunku \ref{fig:uniformCost}, przedstawiony został wynik działania algorytmu Uniform-cost dla zadanego grafu ważonego i skierowanego.

\paragraph{}
\begin{figure}[!h]
 \centering
 \includegraphics{algorithms/uniformCost}
 \caption{Wynik działania algorytmu Dijkstry na przykładowym skierowanym, ważonym grafie}
 \label{fig:uniformCost}
\end{figure}


\subsection{Floyd-Warshall}
\paragraph{}
Algorytm Floyda-Warshalla służy do znajdowania ścieżek z największym przepływem pomiędzy wszystkimi parami wierzchołków w grafie ważonym.
Pozwala on na szukanie ścieżek zarówno w grafach z dodatnimi jak i ujemnymi wagami krawędzi, lecz nie mogą wystąpić cykle o ujemnej wadze.
W naszej implementacji wykorzystujemy ten algorytm do wybrania ścieżki pomiędzy wierzchołkiem początkowym i końcowym bazującej na przepływie.

\paragraph{}
Jak widać na rysunku \ref{fig:floyd}, algorytm Floyda-Warshalla wyznaczył poprawną ścieżkę w zadanym grafie.
\begin{figure}[!h]
 \centering
 \includegraphics{algorithms/floyd}
 \caption{Wynik działania algorytmu Floyda-Warshalla na przykładowym grafie}
 \label{fig:floyd}
\end{figure}

\subsection{Prim}
\paragraph{}
Algorytm Prima oryginalnie służy do wyznaczania minimalnego drzewa rozpinającego (MST), można go jednak zaadopotować również do wyszukiwania najkrótszych ścieżek w grafie.
Wystarczy po znalezieniu minimalnego drzewa rozpinającego wyszukać w nim ścieżkę pomiędzy interesującymi nas punktami za pomocą innego, szybkiego algorytmu.
Algorytm został wynaleziony w 1930 przez czeskiego matematyka Vojtěcha Jarníka, a następnie odkryty na nowo przez informatyka Roberta C. Prima w 1957 oraz niezależnie przez Edsgera Dijkstrę w 1959. 
Z tego powodu algorytm nazywany jest również czasami algorytmem Dijkstry–Prima, algorytmem DJP, algorytmem Jarníka, albo algorytmem Prima–Jarníka.

\paragraph{}
Na rysunku \ref{fig:prim}, przedstawiono wynik działania algorytmu Prima. 
Ścieżka została poprawnie wyznaczone w zadanym nieskierowanym, ważonym grafie.
\begin{figure}[!h]
 \centering
 \includegraphics[width=1.0\textwidth]{algorithms/prim}
 \caption{Wynik działania algorytmu Prima na przykładowym nieskierowanym, ważonym grafie}
 \label{fig:prim}
\end{figure}

\subsection{Kruskal}
\paragraph{}
Algorytm Kruskala, podobnie jak algorytm Prima, służy do wyznaczania minimalnego drzewa rozpinającego dla grafu nieskierowanego i ważonego. 
Z wyznaczonego drzewa można następnie wyznaczyć ścieżkę pomiędzy dwoma wierzchołkami.
Innymi słowy, znajduje drzewo zawierające wszystkie wierzchołki grafu, którego waga jest najmniejsza możliwa. 
Jest to przykład algorytmu zachłannego.
Został on po raz pierwszy opublikowany przez Josepha Kruskala w 1956 roku.

\paragraph{}
Jak widać na rysunku \ref{fig:kruskal}, algorytm Kruskala wyznaczył poprawną ścieżkę w zadanym nieskierowanym, ważonym grafie.
\begin{figure}[!h]
 \centering
 \includegraphics{algorithms/kruskal}
 \caption{Wynik działania algorytmu Kruskala na przykładowym nieskierowanym, ważonym grafie}
 \label{fig:kruskal}
\end{figure}

\section{Metryki}

Przygotowany przez nas serwis udostępnia 3 rodzaje meta-informacji związanych z wykonywaniem zadania planowania:
\begin{itemize}
	\item Mierzenie czasu wykonania algorytmu.
	\item Mierzenie maksymalnego zużycia pamięci.
	\item Sumaryczną wagę ścieżki.
\end{itemize}

Metryki te pozwalają ocenić zarówno koszt wykonania obliczeń jak i jakość uzyskanych wyników.

\subsection{Czas wykonania}

W celu mierzenia czasu pracy algorytmu obliczamy różnicę pomiędzy czasem, w którym algorytm został uruchomiony a czasem, w którym się zakończył.
Nie jest to sposób idealny - nie bierze pod uwagę operacji przełączania kontekstu przez planera systemowego.
W efekcie może się okazać, że przy dużym obciążeniu węzła obliczeniowego czasy wykonania mogą zostać zafałszowane.
Nie istnieje jednak uniwersalny - niezależny od systemu operacyjnego - sposób zebrania takich pomiarów.

\subsection{Zużycie pamięci}

W celu pobrania informacji o maksymalnym zużyciu pamięci przez algorytm dokonujemy jego instrumentacji, poprzez uruchomienie wątku agregującego kolejne rozmiary sterty.
Warto mieć na uwadze, że może to wpłynąć na pomiary czasu wykonania algorytmu.
Ponieważ Java jest językiem z zarządzaną pamięcią, pozyskane wyniki nie są do końca obiektywne - istnieje szansa, że algorytm mógłby działać z mniejszym rozmiarem sterty niż ten podany przez nasz algorytm.
Wynika to z faktu, że nie mamy bezpośredniego wpływu na działanie Garbage Collectora.
W efekcie część obiektów może już nie być potrzebna, ale jednocześnie nie zostać zwolniona.
Podawaną przez nas informację należy więc traktować jako górne ograniczenie pamięci - algorytm na pewno nie będzie wymagał jej więcej, ale może wymagać znacznie mniej.

\subsection{Waga ścieżki}

W celu oszacowania jakości wybranego algorytmu dokonujemy pomiaru długości / sumarycznej wagi wyznaczonej ścieżki.
Dzięki temu użytkownik ma możliwość porównać ze sobą wyniki uzyskane dla tego samego grafu, dla różnych algorytmów i stwierdzić, który algorytm daje rozwiązanie lepsze.

\section{Enterprise Integration Patterns}
W naszym projekcie wykorzystaliśmy cztery różne biblioteki do obliczeń grafowych. 
Każda z nich posiadała inną reprezentację danych. Wykorzystane biblioteki to:

\begin{itemize}
 \item JUNG
 \item GeoTools
 \item aima-java
 \item JGraphT
\end{itemize}

Aby umożliwić wykorzystanie algorytmów pochodzących z tych bibliotek, potrzebne było dokonanie transformacji przychodzących danych wejściowych z
formatu GEXF do formatu wykorzystywanego przez daną bibliotekę.
W tym celu skorzystaliśmy ze wzorców projektowych Enterprise Integration Patterns zaproponowanych w \cite{hohpe2004enterprise}.

\subsection{Normalizer}
Wzorzec projektowy Normalizer wykorzystuje Message Router aby skierować przychodzącą wiadomość do odpowiadającego jej Message Translatora.
Wymaga to, aby Message Router wykrył typ przesłanej wiadomości i na tej podstawie skierował ją w odopowienie miejsce.
Sytuacja została przedstawiona na rysunku \ref{fig:normalizer}

\begin{figure}[!h]
 \centering
 \includegraphics[width=1.0\textwidth]{eip/Normalizer}
 \caption{Wzorzec Normalizer}
 \label{fig:normalizer}
\end{figure}

\subsection{Message Router}
Worzec Message Router zapewnia przekazanie przychodzącej wiadomości do odpowiedniej kolejki w której odbędzie się dalsze przetwarzanie wiadomości.
Dokonuje się to, na podstawie wykrytego typu wiadomości.
Sytuacja została przedstawiona na rysunku \ref{fig:messageRouter}

\begin{figure}[!h]
 \centering
 \includegraphics[width=1.0\textwidth]{eip/MessageRouter}
 \caption{Wzorzec Message Router}
 \label{fig:messageRouter}
\end{figure}

\subsection{Message Translator}
Wzorzec projektowy Message Translator wykorzystywany jest do transformacji przychodzącej wiadomości do określonego formatu wyjściowego.
Symbolicznie zostało to przedstawione na rysunku \ref{fig:messageTranslator}

\begin{figure}[!h]
 \centering
 \includegraphics{eip/MessageTranslator}
 \caption{Wzorzec Message Translator}
 \label{fig:messageTranslator}
\end{figure}

\section{Uruchamianie zadań}

Zadania obliczeniowe uruchamiane są asynchronicznie, w osobnych instancjach Wirtualnej Maszyny Java.
Odbywa się to za pomocą narzędzia Maven, które startuje moduł {\it executor} podając mu jako argument uruchomienia lokalizację tymczasowego pliku z parametrami.
W związku z tym obliczenia mogą być z powodzeniem wykonywane zdalnie - konieczne jest jedynie wywołanie polecenia

{\it mvn exec:java}

Dla modułu {\it executor} poprzez protokół SSH.

Wykorzystanie Mavena do uruchamiania zadań powoduje pewne narzuty czasowe, które warto mieć na uwadze.
Start narzędzia oraz przetwarzanie pliku konfiguracyjnego może trwać nawet do kilku sekund i w związku z tym nawet przy przesyłaniu bardzo małego zadania obliczeniowego należy uzbroić się w cierpliwość.
Czas startowania Mavena nie ma wpływu na zwracany czas działania algorytmu, ponieważ pomiar czasu zaczyna się już po uruchomieniu aplikacji.

Ponieważ Maven jest konieczny nie tylko do zbudowania aplikacji, ale także do uruchamiania zadań, jest to niezbędny komponent.
O ile proces budowania jest możliwy do przeprowadzenia ręcznie (chociaż jest to nieporównywalnie bardziej skomplikowane), o tyle uruchomienie obliczeń bez jego użycia wymagałoby dość sporych zmian w implementacji.
