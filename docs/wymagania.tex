\chapter{Wymagania}

Zasięg projektu został na samym początku jasno określony przez prowadzącego.
Ustalone zostały zarówno wymagania dotyczące dostarczanych funkcjonalnści jak i wymagania dotyczące jakości i ograniczeń, w ramach których system powinien działać.

\section{Wymagania funkcjonalne}

Podstawowe funkcjonalności, którymi powinien charakteryzować się system:

\begin{itemize}
	\item Udostępnienie serwisu umożliwiajacego planowanie ścieżek w zadanych grafach.
	\item Umożliwienie wyboru przynajmniej 5 różnych algorytmów szukania ścieżek.
	\item Umożliwienie asynchronicznego wykonywania obliczeń.
	\item Zwracanie klientowi statystyk oraz wyników planowania.
\end{itemize}

\section{Wymagania niefunkcjonalne}

Podstawowe wymogi dotyczące jakości tworzonego produktu:

\begin{itemize}
	\item System powinien umożliwiać wykonywanie wielu zadań jednocześnie.
	\item System powinien umożliwiać klientowi uzyskanie za pomocą znanego interfejsu informacji o udostępnianych funkcjonalnościach.
	\item System powinien charakteryzować się niskim zużyciem zasobów, tak aby większość mocy obliczeniowej oraz pamięci mogła być przeznaczona na realizację zadań planowania.
\end{itemize}