\documentclass[pdflatex,11pt]{dokumentacja}
\usepackage[polish]{babel}
\usepackage[utf8]{inputenc}

% dodatkowe pakiety
\usepackage{enumerate}
\usepackage{listings}
\usepackage{hyperref}
\usepackage{fancyvrb}
\usepackage{color}
\usepackage{float}

\floatstyle{ruled}
\newfloat{program}{thp}{lop}
\floatname{program}{Kod źródłowy}
\newenvironment{code}{\selectfont\small}{\par}

\lstloadlanguages{TeX}
\hypersetup{%
    pdfborder = {0 0 0}
}

\newcommand{\listexamplename}{Spis kodów źródłowych} 
\newlistof{example}{exp}{\listexamplename}

\newcommand{\example}[1]{% 
      \refstepcounter{example} 
      \par\noindent\caption{#1} 
      \addcontentsline{exp}{example} 
      {\protect\numberline{\theexample}#1}\par 
}

%---------------------------------------------------------------------------

\author{Paweł Kaczmarczyk, Stanisław Podgórski}
\shortauthor{P. Kaczmarczyk, S. Podgórski}

\titlePL{Rozproszony system do testowania algorytmów planowania: Serwisy wykonujące planowanie przy pomocy różnych algorytmów}

\shorttitlePL{Serwisy wykonywujące planowanie przy pomocy różnych algorytmów}

\thesistypePL{Zaawansowane techniki integracji systemów}

\date{2013}

\departmentPL{Katedra Informatyki}

\facultyPL{Wydział Informatyki, Elektroniki i Telekomunikacji}

\setlength{\cftsecnumwidth}{10mm}

%---------------------------------------------------------------------------

\begin{document}

\titlepages

\include{abstract}

\tableofcontents
\newpage
\listoffigures 
\newpage
\listofexample

\clearpage

\chapter{Cel projektu}

Celem realizowanego projektu było:

\begin{itemize}
	\item Udostepnienie serwisu odszukiwania scieżek w grafach.
	\item Umożliwienie wyboru różnych algorytmów planowania.
	\item Udostępnienie statystyk do ewaluacji dostarczanych algorytmów.
\end{itemize}

Serwis miał otrzymywać zlecenia w postaci grafu oraz węzła początkowego i końcowego, a następnie zwracać ścieżkę pomiędzy tymi punktami oraz meta-informacje związane z procesem obliczania ścieżki.

\section{Poszukiwanie ścieżek}

Problem poszukiwania możliwie optymalnej ścieżki w grafie sprowadza się do wyznaczenia zbioru węzłów lub zbioru krawędzi (istotne dla multigrafów) poprzez które można dostać się z węzła początkowego do węzła końcowego.
Interesują nas zwykle ścieżki, dla których sumaryczna wartość wag mijanych krawędzi będzie najmniejsza.
Rysunek \ref{fig:planowanie} przedstawia działanie algorytmu planowania dla przykładowego garfu.

\begin{figure}[!h]
	\centering
	\includegraphics{img/planowanie.png}
	\caption{Przykład wyszukiwania ścieżki w grafie.}
	\label{fig:planowanie}
\end{figure}

\section{Motywacja projektu}

Stworzony system może zostać wykorzystany w kilku różnych scenariuszach użycia:
\begin{itemize}
	\item Ewaluacja działania różnych algorytmów dla wybranego typu grafów.
	\item Ewaluacja wybranego algorytmu planowania dla różnych rodzajów grafów.
	\item Określenie wymagań czasowych i pamięciowych dla konkretnych problemów planowania
\end{itemize}
\chapter{Wymagania}

Zasięg projektu został na samym początku jasno określony przez prowadzącego.
Ustalone zostały zarówno wymagania dotyczące dostarczanych funkcjonalnści jak i wymagania dotyczące jakości i ograniczeń, w ramach których system powinien działać.

\section{Wymagania funkcjonalne}

Podstawowe funkcjonalności, którymi powinien charakteryzować się system:

\begin{itemize}
	\item Udostępnienie serwisu umożliwiajacego planowanie ścieżek w zadanych grafach.
	\item Umożliwienie wyboru przynajmniej 5 różnych algorytmów szukania ścieżek.
	\item Umożliwienie asynchronicznego wykonywania obliczeń.
	\item Zwracanie klientowi statystyk oraz wyników planowania.
\end{itemize}

\section{Wymagania niefunkcjonalne}

Podstawowe wymogi dotyczące jakości tworzonego produktu:

\begin{itemize}
	\item System powinien umożliwiać wykonywanie wielu zadań jednocześnie.
	\item System powinien umożliwiać klientowi uzyskanie za pomocą znanego interfejsu informacji o udostępnianych funkcjonalnościach.
	\item System powinien charakteryzować się niskim zużyciem zasobów, tak aby większość mocy obliczeniowej oraz pamięci mogła być przeznaczona na realizację zadań planowania.
\end{itemize}
\chapter{Wizja}

Przyjęta na początku wizja aplikacji zakładała:

\begin{itemize}
	\item Możliwość obsługiwania wielu klientów jednocześnie.
	\item Potencjalną możliwość uruchamiania zadań planowania na zdalnych węzłach obliczeniowych.
	\item Zwracanie wyników planowania bezpośrednio do zleceniodawcy przez węzeł obliczeniowy, który w tym celu informował by nas o adresie, na który odsyłać wyniki.
\end{itemize}

Rusynek \ref{fig:wizja} przedstawia schemat osadzenia tworzonego serwisu w jego potencjalnych środowisku uruchomieniowym.
W realnej sytuacji oczywiście potencjalnych klientów może być $m$ a węzłów obliczeniowych $n$.

\begin{figure}[!h]
	\centering
	\includegraphics[width=\linewidth]{img/wizja}
	\caption{Wstępna wizja aplikacji i jej otoczenia.}
	\label{fig:wizja}
\end{figure}

\include{xmlstyle}
\chapter{Implementacja}

\section{Architektura}

\section{Web serwisy}

\section{Format danych}

\section{Algorytmy}

\section{Metryki}

\section{EIP}

\section{Uruchamianie zadań}

\chapter{Instrukcja}

\section{Budowanie projektu}

\section{Uruchomienie projektu}

\nocite{*}

\bibliographystyle{plain}
\bibliography{bibliografia}

\end{document}
